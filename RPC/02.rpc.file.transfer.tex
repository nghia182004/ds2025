\documentclass[a4paper,12pt]{article}

\usepackage{geometry}
\usepackage{verbatim}
\usepackage{listings}
\usepackage{graphicx}

\geometry{margin=1in}

\title{\textbf{1-1 RPC File Transfer System}}
\author{}
\date{}

\begin{document}

\maketitle

\section*{Introduction}

This report is about the design, implementation, and organization of a File Transfer System upgraded to use Remote Procedure Call (RPC). The system enables a client to send a file to a server, where the server processes and saves the file content. XML-RPC is used to handle remote communication.

\section*{Protocol Design}
The RPC-based file transfer protocol involves the following steps:
\begin{enumerate}
    \item The server starts and listens for RPC calls.
    \item The client connects to the server via XML-RPC.
    \item The client encodes the file content in Base64 format and calls a remote function on the server to send the file.
    \item The server decodes the received data and saves it to a local file.
    \item Both components complete the operation without requiring a direct TCP socket connection.
\end{enumerate}

\subsection*{Protocol Flow}
The communication protocol flow is represented below:
\begin{verbatim}
+----------+               +----------+
|  Client  |               |  Server  |
+----------+               +----------+
     |                           |
     |   RPC Call: receive_file  |
     |-------------------------->|
     |                           |
     |   Send file data          |
     |-------------------------->|
     |                           |
+----------+                     |
|  Close   |                     |
+----------+                     |
\end{verbatim}

\section*{System Organization}
The system is organized into two main components:
\begin{enumerate}
    \item \textbf{Client:} Reads a file, encodes its content, and sends it to the server using an XML-RPC call.
    \item \textbf{Server:} Implements an XML-RPC function to receive file data, decode it, and save it locally.
\end{enumerate}

\subsection*{System Architecture}
The architecture is shown below:
\begin{verbatim}
+--------------------+      +--------------------+
| Client             |      | Server             |
|--------------------|      |--------------------|
| XML-RPC Interface  |      | XML-RPC Interface  |
| File Reader        |      | File Writer        |
| Base64 Encoder     |      | Base64 Decoder     |
+--------------------+      +--------------------+
\end{verbatim}

\section*{Implementation}

\subsection*{Client Code Snippet}
\begin{lstlisting}[language=python]
from xmlrpc.client import ServerProxy
import base64

server_url = "http://127.0.0.1:8080"

try:
    proxy = ServerProxy(server_url)
    filename = 'send.txt'

    with open(filename, "rb") as f:
        file_data = base64.b64encode(f.read()).decode('utf-8')

    response = proxy.receive_file(file_data, 'recv.txt')
    print(response)

except Exception as e:
    print(f"An error occurred: {e}")
\end{lstlisting}

\subsection*{Server Code Snippet}
\begin{lstlisting}[language=python]
from xmlrpc.server import SimpleXMLRPCServer
import base64

def receive_file(file_data, filename):
    try:
        with open(filename, "wb") as f:
            f.write(base64.b64decode(file_data))
        return f"File {filename} received successfully."
    except Exception as e:
        return f"Error: {e}"

server = SimpleXMLRPCServer(("127.0.0.1", 8080), allow_none=True)
server.register_function(receive_file, "receive_file")
print("Server is running...")
server.serve_forever()
\end{lstlisting}

\section*{Responsibilities}
\begin{enumerate}
    \item \textbf{Client:}
    \begin{itemize}
        \item Connects to the server via XML-RPC.
        \item Reads the file, encodes it in Base64, and sends it.
    \end{itemize}
    \item \textbf{Server:}
    \begin{itemize}
        \item Accepts RPC calls and processes file transfer requests.
        \item Decodes the file content and saves it locally.
    \end{itemize}
\end{enumerate}

\end{document}
